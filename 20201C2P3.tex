\documentclass[letterpaper,10pt]{article}
\usepackage[utf8]{inputenc}
\usepackage[activeacute,spanish]{babel}
\usepackage{amsmath}
\usepackage{amsfonts}
\usepackage{enumerate}
\usepackage{float}
\usepackage{indentfirst}
\usepackage{graphicx}
\usepackage{url}
\usepackage{multicol}
\usepackage{geometry}
\usepackage{fullpage}
\usepackage{algorithm}
\usepackage{algorithmic}

\usepackage{tikz}
\usetikzlibrary{arrows,petri,topaths,shapes,automata}
\usepackage{tkz-berge}

\usepackage[position=bottom]{subfig}

\setlength\parindent{0pt}

\tikzset{
    %Define standard arrow tip
    >=stealth',
    % Define arrow style
    pil/.style={
           ->,
           thick,}
}

\begin{document}

\thispagestyle{empty}

\begin{minipage}[t]{0.6\textwidth}

{\LARGE \textbf{INF152} Estructuras Discretas}

{\large Profesores: C. Lobos -- M. Bugueño -- G. Treimun}

Universidad T\'ecnica Federico Santa Mar\'{\i}a

Departamento de Inform\'atica -- Junio 10, 2020.

\end{minipage}
\hfill
\begin{minipage}[t]{0.35\textwidth}
%RELLENE CON SUS DATOS PERSONALES:
\textbf{Nombre}: Alan Zapata Silva\\[0.3cm]
\textbf{Rol}: 201956567-2 \textbf{Paralelo}: 201
\end{minipage}

\vspace{0.4cm}

{\Large Certamen 2 -- Pregunta 3 \textbf{40pts.}} 

\vspace{0.4cm}

\begin{enumerate}[1.]
\item La Figura \ref{f:bipartito} muestra un Grafo bipartito desde el conjunto de partida $X$ al conjunto de llegada $Y$, es decir, $G(X\cup Y,E)$ y particular los arcos en ``negrita'' corresponden al Match actual. Aplique el algoritmo de caminos alternantes para encontrar el Match más grande que pueda. Debe dibujar los grafos respectivos y mostrar como se ``alternan'' los arcos, mejorando así el Match [\textbf{10\%}].

\begin{figure}[htb]
\centering
\begin{tikzpicture}[-,auto,semithick,main node/.style={circle,draw}]
\begin{scope} [node distance=1.5cm and 3.0cm]
\node[main node] (x1) {$x_1$};
\node[main node] (x2) [right of=x1] {$x_2$};
\node[main node] (x3) [right of=x2] {$x_3$};
\node[main node] (x4) [right of=x3] {$x_4$};
\node[main node] (x5) [right of=x4] {$x_5$};
\node[main node] (x6) [right of=x5] {$x_6$};
\node[main node] (x7) [right of=x6] {$x_7$};
\node[main node] (x8) [right of=x7] {$x_8$};
\node[main node] (x9) [right of=x8] {$x_9$};
\node[main node] (y1) [below left of=x2, yshift=-0.5cm, xshift=0.4cm] {$y_1$};
\node[main node] (y2) [right of=y1] {$y_2$};
\node[main node] (y3) [right of=y2] {$y_3$};
\node[main node] (y4) [right of=y3] {$y_4$};
\node[main node] (y5) [right of=y4] {$y_5$};
\node[main node] (y6) [right of=y5] {$y_6$};
\node[main node] (y7) [right of=y6] {$y_7$};
\node[main node] (y8) [right of=y7] {$y_8$};
\end{scope}
%conexiones de los estados
\path 	(x1)	edge [dashed]			node {} (y2)
		(x2)	edge [dashed]			node {} (y1)
		      	edge [line width=2.0pt]	node {} (y2)
		(x3)	edge [line width=2.0pt] 	node {} (y1)
		(x3)	edge [dashed]			node {} (y3)
		(x3) 	edge [dashed]			node {} (y4)
		(x3)	edge [dashed]			node {} (y5)
		(x4)	edge [dashed]			node {} (y5)
	    	(x5)	edge [dashed]			node {} (y7)
		(x6)	edge [line width=2.0pt]	node {} (y7)
			edge [dashed]			node {} (y8)
			edge [dashed]			node {} (y5)
		(x7)	edge [dashed]			node {} (y7)
		(x8)	edge [dashed]			node {} (y6)
			edge [dashed]			node {} (y7)
		       	edge [line width=2.0pt]	node {} (y8)
		(x9) 	edge [dashed]			node {} (y8);
\end{tikzpicture}
\caption{Grafo bipartito de $X$ sobre $Y$}
\label{f:bipartito}
\end{figure} \\
\newline
\textbf{Desarrollo:}\\
Ya que nos piden los arboles que pueden mejorar el matching de este grafo, primero identificaremos aquellos nodos que no estan conectados, lo cuales son $x_{1}, x_{4}, x_{5}, x_{6},$ y $x_{9}$.\\
\newline
Comenzaremos con $x_{4}$ al cual le aplicamos el algoritmo de caminos alternantes para intentar encontrar mas Matchs:
\\
\begin{minipage}[t]{0.6\textwidth}


\tikzset{every picture/.style={line width=0.75pt}} %set default line width to 0.75pt        

\begin{tikzpicture}[x=0.75pt,y=0.75pt,yscale=-1,xscale=1]
%uncomment if require: \path (0,590); %set diagram left start at 0, and has height of 590

%Shape: Circle [id:dp1316883711033302] 
\draw   (533,452) .. controls (533,438.19) and (544.19,427) .. (558,427) .. controls (571.81,427) and (583,438.19) .. (583,452) .. controls (583,465.81) and (571.81,477) .. (558,477) .. controls (544.19,477) and (533,465.81) .. (533,452) -- cycle ;
%Shape: Circle [id:dp03181252159067971] 
\draw   (534,362) .. controls (534,348.19) and (545.19,337) .. (559,337) .. controls (572.81,337) and (584,348.19) .. (584,362) .. controls (584,375.81) and (572.81,387) .. (559,387) .. controls (545.19,387) and (534,375.81) .. (534,362) -- cycle ;
%Shape: Circle [id:dp19956870644780278] 
\draw   (605,364) .. controls (605,350.19) and (616.19,339) .. (630,339) .. controls (643.81,339) and (655,350.19) .. (655,364) .. controls (655,377.81) and (643.81,389) .. (630,389) .. controls (616.19,389) and (605,377.81) .. (605,364) -- cycle ;
%Shape: Circle [id:dp878486568276804] 
\draw   (439,364) .. controls (439,350.19) and (450.19,339) .. (464,339) .. controls (477.81,339) and (489,350.19) .. (489,364) .. controls (489,377.81) and (477.81,389) .. (464,389) .. controls (450.19,389) and (439,377.81) .. (439,364) -- cycle ;
%Shape: Circle [id:dp6620971455909709] 
\draw   (272,31) .. controls (272,17.19) and (283.19,6) .. (297,6) .. controls (310.81,6) and (322,17.19) .. (322,31) .. controls (322,44.81) and (310.81,56) .. (297,56) .. controls (283.19,56) and (272,44.81) .. (272,31) -- cycle ;
%Shape: Circle [id:dp6441516447046123] 
\draw   (350,364) .. controls (350,350.19) and (361.19,339) .. (375,339) .. controls (388.81,339) and (400,350.19) .. (400,364) .. controls (400,377.81) and (388.81,389) .. (375,389) .. controls (361.19,389) and (350,377.81) .. (350,364) -- cycle ;
%Shape: Circle [id:dp964534749720036] 
\draw   (554,266) .. controls (554,252.19) and (565.19,241) .. (579,241) .. controls (592.81,241) and (604,252.19) .. (604,266) .. controls (604,279.81) and (592.81,291) .. (579,291) .. controls (565.19,291) and (554,279.81) .. (554,266) -- cycle ;
%Shape: Circle [id:dp27318155735129535] 
\draw   (434,267) .. controls (434,253.19) and (445.19,242) .. (459,242) .. controls (472.81,242) and (484,253.19) .. (484,267) .. controls (484,280.81) and (472.81,292) .. (459,292) .. controls (445.19,292) and (434,280.81) .. (434,267) -- cycle ;
%Shape: Circle [id:dp7131170895572031] 
\draw   (250,269) .. controls (250,255.19) and (261.19,244) .. (275,244) .. controls (288.81,244) and (300,255.19) .. (300,269) .. controls (300,282.81) and (288.81,294) .. (275,294) .. controls (261.19,294) and (250,282.81) .. (250,269) -- cycle ;
%Shape: Circle [id:dp715446865637286] 
\draw   (134,269) .. controls (134,255.19) and (145.19,244) .. (159,244) .. controls (172.81,244) and (184,255.19) .. (184,269) .. controls (184,282.81) and (172.81,294) .. (159,294) .. controls (145.19,294) and (134,282.81) .. (134,269) -- cycle ;
%Shape: Circle [id:dp43691404053536287] 
\draw   (24,268) .. controls (24,254.19) and (35.19,243) .. (49,243) .. controls (62.81,243) and (74,254.19) .. (74,268) .. controls (74,281.81) and (62.81,293) .. (49,293) .. controls (35.19,293) and (24,281.81) .. (24,268) -- cycle ;
%Shape: Circle [id:dp7155630766688411] 
\draw   (330,173) .. controls (330,159.19) and (341.19,148) .. (355,148) .. controls (368.81,148) and (380,159.19) .. (380,173) .. controls (380,186.81) and (368.81,198) .. (355,198) .. controls (341.19,198) and (330,186.81) .. (330,173) -- cycle ;
%Shape: Circle [id:dp2623615111556197] 
\draw   (199,170) .. controls (199,156.19) and (210.19,145) .. (224,145) .. controls (237.81,145) and (249,156.19) .. (249,170) .. controls (249,183.81) and (237.81,195) .. (224,195) .. controls (210.19,195) and (199,183.81) .. (199,170) -- cycle ;
%Shape: Circle [id:dp6902163593991683] 
\draw   (271,108) .. controls (271,94.19) and (282.19,83) .. (296,83) .. controls (309.81,83) and (321,94.19) .. (321,108) .. controls (321,121.81) and (309.81,133) .. (296,133) .. controls (282.19,133) and (271,121.81) .. (271,108) -- cycle ;
%Straight Lines [id:da7166375334718627] 
\draw  [dash pattern={on 0.84pt off 2.51pt}]  (277,124) -- (244,152) ;
%Straight Lines [id:da6489321485873476] 
\draw  [dash pattern={on 0.84pt off 2.51pt}]  (338,152) -- (316,126) ;
%Straight Lines [id:da24537085179388096] 
\draw  [dash pattern={on 0.84pt off 2.51pt}]  (439,250) -- (374,190) ;
%Straight Lines [id:da2901596700907707] 
\draw  [dash pattern={on 0.84pt off 2.51pt}]  (542,343) -- (481,284) ;
%Straight Lines [id:da5504371474513357] 
\draw  [dash pattern={on 0.84pt off 2.51pt}]  (630,339) -- (594,286) ;
%Straight Lines [id:da6884628330266744] 
\draw  [dash pattern={on 0.84pt off 2.51pt}]  (440,287) -- (393,343) ;
%Shape: Circle [id:dp9509170977400759] 
\draw   (28,522) .. controls (28,508.19) and (39.19,497) .. (53,497) .. controls (66.81,497) and (78,508.19) .. (78,522) .. controls (78,535.81) and (66.81,547) .. (53,547) .. controls (39.19,547) and (28,535.81) .. (28,522) -- cycle ;
%Shape: Circle [id:dp45437699552433275] 
\draw   (25,445) .. controls (25,431.19) and (36.19,420) .. (50,420) .. controls (63.81,420) and (75,431.19) .. (75,445) .. controls (75,458.81) and (63.81,470) .. (50,470) .. controls (36.19,470) and (25,458.81) .. (25,445) -- cycle ;
%Shape: Circle [id:dp2686356300954462] 
\draw   (23,359) .. controls (23,345.19) and (34.19,334) .. (48,334) .. controls (61.81,334) and (73,345.19) .. (73,359) .. controls (73,372.81) and (61.81,384) .. (48,384) .. controls (34.19,384) and (23,372.81) .. (23,359) -- cycle ;
%Straight Lines [id:da7059597585966062] 
\draw  [dash pattern={on 0.84pt off 2.51pt}]  (50,470) -- (53,497) ;
%Straight Lines [id:da3019561066433506] 
\draw  [dash pattern={on 0.84pt off 2.51pt}]  (49,293) -- (48,334) ;
%Straight Lines [id:da4606434618639411] 
\draw  [dash pattern={on 0.84pt off 2.51pt}]  (240,194) -- (268,247) ;
%Straight Lines [id:da8793127641477738] 
\draw  [dash pattern={on 0.84pt off 2.51pt}]  (205,186) -- (165,243) ;
%Straight Lines [id:da7461194383564094] 
\draw    (71,257) -- (203,177) ;
%Straight Lines [id:da14705008012717158] 
\draw    (296,83) -- (297,56) ;
%Straight Lines [id:da4236034518664148] 
\draw    (50,420) -- (48,384) ;
%Straight Lines [id:da017058246587106973] 
\draw    (464,339) -- (459,292) ;
%Straight Lines [id:da19594019047740363] 
\draw    (558,427) -- (559,387) ;
%Straight Lines [id:da6169846248265834] 
\draw  [dash pattern={on 0.84pt off 2.51pt}]  (564,336) -- (573,290) ;
%Straight Lines [id:da40612120953564934] 
\draw    (380,173) -- (558,249) ;

% Text Node
\draw (293,28) node [anchor=north west][inner sep=0.75pt]   [align=left] {x4\\};
% Text Node
\draw (553,357) node [anchor=north west][inner sep=0.75pt]   [align=left] {x8\\};
% Text Node
\draw (626,354) node [anchor=north west][inner sep=0.75pt]   [align=left] {x9\\};
% Text Node
\draw (370,359) node [anchor=north west][inner sep=0.75pt]   [align=left] {x5\\};
% Text Node
\draw (459,352) node [anchor=north west][inner sep=0.75pt]   [align=left] {x7\\};
% Text Node
\draw (354,162) node [anchor=north west][inner sep=0.75pt]   [align=left] {x6\\};
% Text Node
\draw (219,163) node [anchor=north west][inner sep=0.75pt]   [align=left] {x3\\};
% Text Node
\draw (40,356) node [anchor=north west][inner sep=0.75pt]   [align=left] {x2\\};
% Text Node
\draw (43,513) node [anchor=north west][inner sep=0.75pt]   [align=left] {x1\\};
% Text Node
\draw (294,96) node [anchor=north west][inner sep=0.75pt]   [align=left] {y5\\};
% Text Node
\draw (553,440) node [anchor=north west][inner sep=0.75pt]   [align=left] {y9\\};
% Text Node
\draw (574,254) node [anchor=north west][inner sep=0.75pt]   [align=left] {y8\\};
% Text Node
\draw (454,255) node [anchor=north west][inner sep=0.75pt]   [align=left] {y7\\};
% Text Node
\draw (268,259) node [anchor=north west][inner sep=0.75pt]   [align=left] {y4\\};
% Text Node
\draw (152,261) node [anchor=north west][inner sep=0.75pt]   [align=left] {y3\\};
% Text Node
\draw (42,257) node [anchor=north west][inner sep=0.75pt]   [align=left] {y1\\};
% Text Node
\draw (47,435) node [anchor=north west][inner sep=0.75pt]   [align=left] {y2\\};


\end{tikzpicture}

\captionof{figure}{$x_{4}$}
\end{minipage}\\
\newline
En esta representacion de arbol, vemos como tiene 6 elementos de $x$ unidos a elementos de $y$.\\
\newline






Ahora veremos el caso de $x_{5}$ al cual le aplicamos el algoritmo de caminos alternantes para intentar encontrar mas Matchs:
\\
\newline
\begin{minipage}[t]{0.6\textwidth}

\captionof{figure}{$x_{}$}
\end{minipage}
En esta representacion de arbol, vemos como tiene 6 elementos de $x$ unidos a elementos de $y$.\\
\newline



Ahora veremos el caso de $x_{5}$ al cual le aplicamos el algoritmo de caminos alternantes para intentar encontrar mas Matchs:
\\
\newline
\begin{minipage}[t]{0.6\textwidth}

\captionof{figure}{$x_{}$}
\end{minipage}
En esta representacion de arbol, vemos como tiene 6 elementos de $x$ unidos a elementos de $y$.\\
\newline



Ahora veremos el caso de $x_{5}$ al cual le aplicamos el algoritmo de caminos alternantes para intentar encontrar mas Matchs:
\\
\newline
\begin{minipage}[t]{0.6\textwidth}

\captionof{figure}{$x_{}$}
\end{minipage}
En esta representacion de arbol, vemos como tiene 6 elementos de $x$ unidos a elementos de $y$.\\
\newline



Ahora veremos el caso de $x_{5}$ al cual le aplicamos el algoritmo de caminos alternantes para intentar encontrar mas Matchs:
\\
\newline
\begin{minipage}[t]{0.6\textwidth}

\captionof{figure}{$x_{}$}
\end{minipage}
En esta representacion de arbol, vemos como tiene 6 elementos de $x$ unidos a elementos de $y$.\\
\newline



Ahora veremos el caso de $x_{5}$ al cual le aplicamos el algoritmo de caminos alternantes para intentar encontrar mas Matchs:
\\
\newline
\begin{minipage}[t]{0.6\textwidth}

\captionof{figure}{$x_{}$}
\end{minipage}






\item Utilizando la fórmula de \textbf{deficiencia}, demuestre que el Match encontrado en la pregunta 1 es maximal. \textbf{Nota}: puede calcular la deficiencia utilizando todos los $A\subseteq X$, es decir, para los $2^9 = 512$ subconjuntos. O bien, puede utilizar el formalismo aprendido en técnicas de demostración y con un argumento, que incluya el cálculo de la deficiencia, demostrar que el Match encontrado es maximal [\textbf{10\%}]. 







\item Imagínese que se dibuja un grafo sobre un cilindro, de forma tal que todos los nodos estarán equidistantes con su vecinos horizontales y verticales. Cada nodo se conectará con sus 2 vecinos horizontales, sus 2 verticales y sus 4 vecinos en diagonal, salvo por supuesto los nodos que estén en los extremos inferior y superior del cilindro, que tendrán solo 5 conexiones. Cada ``fila'' de nodos en el cilindro tiene la misma cantidad de nodos y entre ellos forman un ciclo. No se sabe cuantos nodos hay, solo que se respetan las restricciones mencionadas. Determine el número crómatico para cualquier cantidad de nodos que cumplan con las restricciones dadas [\textbf{20\%}]. \textbf{Nota}: puede analizar por casos. 
\end{enumerate}

\end{document}



